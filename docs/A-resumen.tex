\cleardoublepage
\phantomsection
\chapter*{Resumen}
\addcontentsline{toc}{chapter}{Resumen}
En los tiempos que corren, la interacción con sistemas automáticos es una parte fundamental a la hora de realizar tareas cotidianas en sistemas informáticos. Realizar una gestión sobre un pedido en una tienda online o solicitar asistencia técnica son algunas de las aplicaciones suelen recaer en automatismos para ofrecer una respuesta rápida, acotada y certera. En este contexto, los llamados Chatbots, consistentes en una inteligencia artificial capaz de simular el comportamiento conversacional de una persona, permiten ofrecer una interfaz amigable para automatizar tareas. El objetivo de este TFM es el de desarrollar un caso de estudio de la aplicación de un chatbot para llevar a cabo el servicio de atención al consumidor. El objetivo de este TFM es estudiar el panorama de los chatbot en la actualidad, mostrando su evolución desde sus inicios hasta el día de hoy comparando sus características y mostrar un caso práctico implantando uno de ellos. Para profundizar en el estudio, se va a desarrollar un caso práctico de uso del chatbot de código libre RASA, haciendo uso de técnicas de machine learning para entrenar sus modelos y comparando los resultados obtenidos mediante la conversación con el robot. Se pretende con ello ver las posibilidades de evolución de un chatbot al involucrar aprendizaje en el modelo de conversación del robot, y comparar los resultados obtenidos con los distintos algoritmos utilizados. Para ello se utilizará Jupyter Notebook y Python.
\vfill
\textbf{Palabras clave:} Chatbot, RASA, Machine Learning, Python.