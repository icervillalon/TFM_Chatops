\cleardoublepage
\phantomsection
\chapter*{Abstract}
\addcontentsline{toc}{chapter}{Abstract}
Nowadays, interaction with automatic systems is a fundamental part in order to perform day-to-day tasks in informatic systems. Managing an online shop order or asking for technical assistance are some of the applications where automatisms can offer a fast and precise response. In this context, the so-called Chatbots, that consist in an AI capable of simulating an human-like conversation, can offer a friendly user interface to automatize these tasks. This TFM objective is to develop a case of study of a chatbot applicated to a customer service. In order to do this, a practical case of a open-source code chatbot called RASA is proposed, using machine learning techniques to train its models, comparing the results obtained in a conversation with the bot. The objective is to check the possibilities of evolution that chatbot when machine learning in the conversational model is involved, and compare the results of different machine learning algorithms. The development will be done using RASA as the chatbot where we are going to develop the use case, Jupyter Notebook to make human-friendly code and keep the results organized, Python 3.8 to run the algorithms, notebooks and RASA. 
\vfill
\textbf{Keywords: }Chatbot, RASA, Machine Learning, Python 